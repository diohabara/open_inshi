% class
\documentclass[a4paper,12pt,xelatex,ja=standard]{bxjsarticle}

% packages
%% mathematical notations
\usepackage{amsthm,amsmath,amssymb,amsfonts} % mathematical notations
\usepackage{bm} % bold character
\usepackage{latexsym} % more mathematical notations
\usepackage{physics} % physical notations
%% graphs
\usepackage{graphicx, xcolor} % graph
\usepackage{circuitikz} % for circuit elements
\usepackage{float} % positioning of graphs
\usepackage{siunitx} % SI units
\usepackage{tikz} % graphic elements
\usepackage{wrapfig} % must be after float package.
%% type system
\usepackage{bussproofs} % proof tree
%% code
\usepackage[ruled,vlined]{algorithm2e} % pseudo code
\usepackage{listings} % source code

% Basic information
\title{電子情報学専攻 \, 専門 \\ 令和元年 \, 解答・解説}
\author{diohabara}
\date{\today}

\begin{document}
\maketitle

\section*{第1問\ 電気・電子回路}

\subsection*{(1)}

% ref: https://jeea.or.jp/course/contents/01133/

スイッチを短絡させると、以下の回路方程式が成り立つ。

\[
  L \frac{d i(\tau)}{d \tau} = v(t)
\]

電源は定電圧電源なので \(v(t) = v(0) = E\) であり、 \(\tau = [0, t]\) 上で \(\tau\) に関して積分して整理すると

\[
  i(t) - i(0) = \frac{E}{L}t
\]

問題文より $t = 0$ で $i(0) = 0$ だから

\[
  i(t) = \frac{E}{L}t
\]

\subsection*{(2)}

スイッチを開放すると、 \(t = [T_0, T_0 + T_1)\) において以下の回路方程式が成り立つ。

\[
  L \frac{d i(t)}{d t} + \frac{1}{C}\int^{t}_{T_0}i(\tau) d \tau = E
\]

両辺を微分して整理すると

\begin{equation*}
  \begin{split}
    0 &= L \frac{d^2 i(t)}{dt^2} + \frac{1}{C}i(t) \\
    \frac{d^2 i(t)}{dt^2} &= - \frac{1}{LC} i(t)
  \end{split}
\end{equation*}

よって、\(i(t)\)の一般解は

\[
  i(t) = A \sin{\frac{1}{\sqrt{LC}}}t+ B \cos{\frac{1}{\sqrt{LC}}}t
\]

と書ける。

(1)より\(i(T_0) = \frac{E}{L}T_0\)であり\(\left.\frac{di(t)}{dt}\right|_{t=T_0}=0\)

\(\frac{di(t)}{dt} = \frac{A}{\sqrt{LC}}\cos \frac{t}{\sqrt{LC}} - \frac{B}{\sqrt{LC}} \sin \frac{t}{\sqrt{LC}}\)より

\[
  \left.\frac{di(t)}{dt}\right|_{t=T_0} = \frac{A}{\sqrt{LC}}\cos \frac{T_0}{\sqrt{LC}} - \frac{B}{\sqrt{LC}} \sin \frac{T_0}{\sqrt{LC}} = 0
\]

よって

\[
  A = B \tan \frac{T_0}{\sqrt{LC}}
\]

また

\begin{equation*}
  \begin{split}
    i(T_0)
    &= \frac{A}{\sqrt{LC}}\cos \frac{T_0}{\sqrt{LC}} - \frac{B}{\sqrt{LC}} \sin \frac{T_0}{\sqrt{LC}} \\
    &= B \left( \sin{\frac{T_0}{\sqrt{LC}}} \tan{\frac{T_0}{\sqrt{LC}}} + \cos{\frac{T_0}{\sqrt{LC}}} \right) \\
    &= B \frac{\sin^2{\frac{T_0}{\sqrt{LC}}} + \cos^2{\frac{T_0}{\sqrt{LC}}}}{\cos{\frac{T_0}{\sqrt{LC}}}} \\
    &= \frac{B}{\cos{\frac{T_0}{\sqrt{LC}}}}
  \end{split}
\end{equation*}

よって、\(i(T_0) = \frac{E}{L}T_0\)より

\begin{equation*}
  \begin{split}
    B &= \frac{E}{L}T_0 \cos{\frac{T_0}{\sqrt{LC}}} \\
    A &= B \tan{\frac{T_0}{\sqrt{LC}}} = \frac{E}{L} T_0 \sin{\frac{T_0}{\sqrt{LC}}}
  \end{split}
\end{equation*}

よって
\begin{equation*}
  \begin{split}
    i(t)
      &= \frac{E}{L} T_0 \sin{\frac{T_0}{\sqrt{LC}}} \sin{\frac{t}{LC}} + \frac{E}{L} T_0 \cos{\frac{T_0}{\sqrt{LC}}} \cos{\frac{t}{LC}}\\
      &= \frac{E}{L} T_0 \cos{\frac{1}{\sqrt{LC}}} (t - T_0)
  \end{split}
\end{equation*}

これが\(t \leq T_0\)で最初に\(0\)となるのは、\(\frac{1}{\sqrt{LC}}(t - T_0) = \frac{\pi}{2}\)のときなので

\begin{equation*}
  \begin{split}
    &\frac{1}{\sqrt{LC}}((T_0 + T_1) - T_0) = \frac{\pi}{2} \\
    &\therefore T_1 = \frac{\pi \sqrt{LC}}{2}
  \end{split}
\end{equation*}

\subsection*{(3)}

ダイオードがあるため、コンデンサにかかる電圧\(v(t)\)は常に単調増加する。\\
したがって、スイッチを開放しているときにコイルに流れる電流の時間変化

\[
  \frac{di(t)}{dt} = \frac{E - v(t)}{L}
\]

\noindent
は単調減少する。これはつまり、回数を重ねるごとにスイッチ解放後に電流が減少するスピードが早くなるということ。
だから、\(i=0\)となるまでにかかる時間は\(T_1\)からどんどん短くなっていく。\\
よって、各操作でスイッチを開放した後\(T_1\)時間後までに必ず\(i=0\)となっているはずなので、\(i(n(T_1 + T_0)) = 0\)と言える。

\subsection*{(4)}

簡単のため、\(v_k = v(k(T_0 + T_1))\)とおく。

\(k(T_0 + T_1) \leq k (T_0 + T_1) + T_0\)のとき、回路に流れる電流は(1)と同様にして

\begin{equation}
  \begin{split}
    E &= L \frac{d i(t)}{dt}, i(k(T_0 + T_1)) = 0 \\
    \therefore i(t) &= \frac{E}{L}(t - k(T_0 + T_1))
  \end{split}
\end{equation}

よって、\(t = k(T_0 + T_1) + T_0\)のとき\(i(t) = \frac{E}{L}T_0\)である。

\(k(T_0 + T_1) + T_0 \leq t < (k+1) (T_0 + T_1)\)の間について、電源がした仕事とコイル・コンデンサのエネルギーの変化分は等しいので

\[
  E \cdot C(v_{k+1} - v_k) = \left(\frac{1}{2} L \cdot 0^2 + \frac{1}{2}C v_{k+1}^2\right) +
    \left(\frac{1}{2} L (\frac{E}{L}T_0)^2 + \frac{1}{2}C v_{k}^2\right)
\]

\[
  (v_{k+1} + E)^2 = (v_k - E)^2 + \frac{E^2}{LC} {T_0}^2
\]

\(v_0 = v(0) = E\)に注意してこれを解くと

\begin{equation*}
  \begin{split}
    (v_n - E)^2 = n \cdot \frac{E^{2}}{LC}T_0^{2} \\
    \therefore v_n = E \left(1 + T_0 \sqrt{\frac{n}{LC}} \right)
  \end{split}
\end{equation*}

\section*{第2問\ 論理回路}
  \subsection*{(1)}
  AおよびBの最大値、最小値は以下の通り。
  \begin{equation*}
    \begin{split}
      A_{max} &= 0111_{(2)} = 2^3 - 1 = 7_{(10)} \\
      A_{min} &= 1000_{(2)} = -(1000 \oplus 1111 + 1)_{(2)} = -(111 + 1)_{(2)} = -8_{(10)} \\
      B_{max} &= 01_{(2)} = 2^1 - 1 = 1_{(10)} \\
      B_{min} &= 10_{(2)} = -(10 \oplus 11 + 1)_{(2)} = -(1 + 1)_{(2)} = -2_{(10)}
    \end{split}
  \end{equation*}

  \subsection*{(2)}
  半加算器、全加算器は以下のブロックとして用いる。
  \begin{figure}[H]
    \centering
    \includegraphics[width=7cm]{images/ha_and_fa.png}
  \end{figure}

  半加算器は次の通りに構成される。
  \begin{figure}[H]
    \centering
    \includegraphics[width=5cm]{images/half_adder.png}
  \end{figure}

  全加算器は次の通りに構成される。
  \begin{figure}[H]
    \centering
    \includegraphics[width=9cm]{images/full_adder.png}
  \end{figure}

  これらを使って構成できる$A+B$を計算して$Y$を得る回路は次の通り。
  \begin{figure}[H]
    \centering
    \includegraphics[width=11cm]{images/ripple_carry_adder.png}
  \end{figure}

  \subsection*{(3)}
  (2)で設計した回路がオーバフローするとき、$A_3 C_2 = 1$となる。\\
  また、(2)の回路は以下の論理式を満たす。
  \begin{equation*}
    \begin{split}
      &C_2 = A_2 C_1 \\
      &C_1 = A_1 C_0 + A_1 B_1 + B_1 C_0 = C_0(A_1 + B_1) + A_1 B_1 \\
      &C_0 = A_0 B_0
    \end{split}
  \end{equation*}
  これらの式を代入して
  \begin{equation*}
    \begin{split}
      &A_3 A_2 C_1 = 1 \\
      &A_2 A_3 (C_0(A_1 + B_1) + A_1 B_1) = 1 \\
      &A_2 A_3 (A_0 B_0(A_1 + B_1) + A_1 B_1) = 1
    \end{split}
  \end{equation*}
  よって求めるオーバフロー検知機構の回路は次の通り。
  \begin{figure}[H]
    \centering
    \includegraphics[width=11cm]{images/overflow_detector.png}
  \end{figure}

  \subsection*{(4)}
  ビット値の符号を変換する際、それぞれの桁を反転させ$1$を加えれば良い。$A-B$を計算する際は、$A$と、$B$を反転させ1加えたものを(2)の回路で足し合わせれば良い。\\
  よって、$A-B$を計算して$Y$を得る回路は次の通り。
  \begin{figure}[H]
    \centering
    \includegraphics[width=11cm]{images/ripple_carry_subtractor.png}
  \end{figure}

  \subsection*{(5)}
  $1$ビットの信号$D$でオーバフローかどうかを出力し、オーバフロー発生時に$1$、そうでないときに$0$を出力するとする。\\
  (3)と同じように考え、$D=1$となるのは以下の論理式を満たす場合である。
  \begin{equation*}
    \begin{split}
      &A_3 C_2 = 1 \\
      &C_2 = A_2 C_1 \\
      &C_1 = C_0 (A_1 + \bar{B_1}) + A_1 \bar{B_1} \\
      &C_0 = A_0 + \bar{B_0}
    \end{split}
  \end{equation*}
  これを解いて
  \begin{equation*}
    \begin{split}
      &A_3 C_2 = 1 \\
      &A_3 A_2 C_1 = 1 \\
      &A_2 A_3 (C_0(A_1 + \bar{B_1}) + A_1 \bar{B_1}) = 1 \\
      &A_2 A_3 ((A_0 + \bar{B_0})(A_1 + \bar{B_1}) + A_1 \bar{B_1}) = 1 \\
      &A_2 A_3 (A_0A_1 + A_0 \bar{B_1} + A_1\bar{B_0} + \bar{B_0} \bar{B_1}) = 1 \\
      &A_0A_1A_2A_3 + A_0A_2A_3\bar{B_1} + A_1A_2A_3\bar{B_0} + A_2A_3\bar{B_0}\bar{B_1} = 1 \\
    \end{split}
  \end{equation*}

  以上からオーバフローが発生する入力パターンは以下の通り。ただし、*は$0$でも$1$でも良い。
  \begin{center}
    \centering
    \begin{tabular}{|l|l|l|l|l|l|}
    \hline
    $A_0$ & $A_1$ & $A_2$ & $A_3$ & $B_0$ & $B_1$ \\ \hline
    1  & 1  & 1  & 1  & *  & *  \\ \hline
    1  & *  & 1  & 1  & *  & 0  \\ \hline
    *  & 1  & 1  & 1  & 0  & *  \\ \hline
    0  & 0  & 1  & 1  & 0  & 0  \\ \hline
    *  & 1  & 1  & 1  & *  & 0  \\ \hline
    \end{tabular}
  \end{center}

\section*{第3問\ アルゴリズム}
  \subsection*{(1)}
  \subsection*{(2)}
  \subsection*{(3)}
  \subsection*{(4)}

\section*{第4問\ ネットワーク}
  \subsection*{(1)}
  \subsection*{(2)}
  \subsection*{(3)}
  \subsection*{(4)}
  \subsection*{(5)}
    \subsubsection*{(a)}
    \subsubsection*{(b)}

\section*{第5問\ 情報理論}
  \subsection*{(1)}
  \subsection*{(2)}
  \subsection*{(3)}
  \subsection*{(4)}
  \subsection*{(5)}
  \subsection*{(6)}
  \subsection*{(7)}

\end{document}

