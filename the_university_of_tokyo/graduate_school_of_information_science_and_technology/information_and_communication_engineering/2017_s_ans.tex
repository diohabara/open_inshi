% class
\documentclass[a4paper,12pt,xelatex,ja=standard]{bxjsarticle}

% packages
%% mathematical notations
\usepackage{amsthm,amsmath,amssymb,amsfonts} % mathematical notations
\usepackage{bm} % bold character
\usepackage{latexsym} % more mathematical notations
\usepackage{physics} % physical notations
\usepackage{mathtools} % math tools
%% graphs
\usepackage{graphicx, xcolor} % graph
\usepackage{circuitikz} % for circuit elements
\usepackage{float} % positioning of graphs
\usepackage{siunitx} % SI units
\usepackage{tikz} % graphic elements
\usepackage{wrapfig} % must be after float package.
\usepackage{askmaps} % Karnaugh map
%% type system
\usepackage{bussproofs} % proof tree
%% code
\usepackage[ruled,vlined]{algorithm2e} % pseudo code
\usepackage{listings} % source code
\usepackage{inconsolata}
\lstset{
  basicstyle=\footnotesize,
  numbers=left,
  frame={tb}
}
\usetikzlibrary{automata, positioning}
\tikzset{
  ->,
  >={Stealth[round]},
  auto,
  every state/.style={draw}
}

% Basic information
\title{電子情報学専攻 \, 専門 \\ 平成28年 \, 解答・解説}
\author{diohabara}
\date{\today}

\begin{document}
\maketitle

\section*{第1問\ 電気・電子回路}

\section*{第2問\ 計算機アーキテクチャ}
\subsection*{(1)}

\subsection*{(2)}

\subsection*{(3)}

\subsection*{(4)}

\subsection*{(5)}

\section*{第3問\ アルゴリズムとデータ構造}
\subsection*{(1)}
求めるリストLの値の推移は以下の通り。
\begin{table}[]
  \begin{tabular}{|l|l|l|}
  \hline
  \textit{i} & A[i] & L            \\ \hline \hline
  0          & 11   & {11}         \\ \hline
  1          & 10   & {}           \\ \hline
  2          & 11   & {11}         \\ \hline
  3          & 11   & {11, 11}     \\ \hline
  4          & 7    & {11}         \\ \hline
  5          & 11   & {11, 11}     \\ \hline
  6          & 11   & {11, 11, 11} \\ \hline
  7          & 3    & {11, 11}     \\ \hline
  8          & 8    & {11}         \\ \hline
  \end{tabular}
\end{table}

\subsection*{(2)}
$i$番目の要素を処理した後のLの要素の種類数は0または1である。\\
$i+1$番目の要素を処理した後も種類数が0または1であるから帰納法よりLに含まれる要素の種類数は高々1つである。

\subsection*{(3)}
Lに含まれる要素が高々1つであることは(2)で示した。よって、全ての要素を処理した後のLに$u_{MAJORITY}$が含まれることを証明すれば良い。\\
\begin{proof}[証明]
  \[
    \text{count} \coloneqq \text{(Lに含まれている$u_{MAJORITY}$の個数)} - \text{(Lに含まれている$u_{MAJORITY}$以外の要素の個数)}
  \]
  として、処理が終わったときに$0 < count$であることを示す。はじめ、Lは空であるので$count = 0$。A[i]に対して処理を行うとき\\
  \subsubsection*{(i) $A[i] = u_{MAJORITY}$の場合}
  \begin{itemize}
    \item Lに$u_{MAJORITY}$を追加する
    \item Lから$u_{MAJORITY}$以外の要素を1つ取り除く
  \end{itemize}
  のどちらかである。よってcountは1大きくなる
  \subsubsection*{(ii)$A[i] \neq u_{MAJORITY}$の場合}
  \begin{itemize}
    \item LにA[i]を追加する
    \item Lから$u_{MAJORITY}$を1つ取り除く
    \item Lから$u_{MAJORITY}$以外の要素を1つ取り除く
  \end{itemize}
  のいずれかである。
  (i)(ii)からすべての処理が終わったとき
  \[
    \text{(Aに含まれている$u_{MAJORITY}$の個数)} - \text{(Aに含まれている$u_{MAJORITY}$以外の要素の個数)} \leq count
  \]
  であるので、$u_{MAJORITY}$が過半数であることから$0 < count$となる。したがって、過半数を占めるユーザ$u_{MAJORITY}$が存在するとき$u_{MAJORITY}$はLに含まれる唯一のユーザである。
\end{proof}

\subsection*{(4)}
\begin{lstlisting}[language=Python]
u = -1 # Lに含まれるユーザ
counter = 0 # Lに含まれるユーザ数
while True:
    v = read_log()
    if v == -1:
        break
    if counter == 0:
        u = v
        counter = 1
    else:
        if v == u:
            counter += 1
        else:
            counter -= 1
while 0 < counter:
    print(u)
    counter -= 1
\end{lstlisting}

\section*{第4問\ 情報通信}

\section*{第5問\ 情報理論}
\subsection*{(1)}

\subsection*{(2)}

\subsection*{(3)}

\subsection*{(4)}

\subsection*{(5)}

\end{document}
