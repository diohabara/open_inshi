% class
\documentclass[a4paper,12pt,xelatex,ja=standard]{bxjsarticle}

% packages
%% mathematical notations
\usepackage{amsthm,amsmath,amssymb,amsfonts} % mathematical notations
\usepackage{bm} % bold character
\usepackage{latexsym} % more mathematical notations
\usepackage{physics} % physical notations
\usepackage{mathtools} % math tools
%% graphs
\usepackage{graphicx, xcolor} % graph
\usepackage{circuitikz} % for circuit elements
\usepackage{float} % positioning of graphs
\usepackage{siunitx} % SI units
\usepackage{tikz} % graphic elements
\usepackage{wrapfig} % must be after float package.
\usepackage{askmaps} % Karnaugh map
%% type system
\usepackage{bussproofs} % proof tree
%% code
\usepackage[ruled,vlined]{algorithm2e} % pseudo code
\usepackage{listings} % source code
\usepackage{inconsolata}
\lstset{
  basicstyle=\footnotesize,
  numbers=left,
  frame={tb}
}
\usetikzlibrary{automata, positioning, calc, quotes}
\tikzset{
  ->,
  >={Stealth[round]},
  auto,
  every state/.style={draw},
  node distance=3cm
}
\newcommand\DoubleLine[5][4pt]{%
    \path(#2)--(#3)coordinate[at start](h1)coordinate[at end](h2);
    \draw[<-,very thick,black] ($(h1)!#1!90:(h2)$)  to ["#4"]   ($(h2)!#1!-90:(h1)$);
    \draw[->,very thick,  red] ($(h1)!#1!-90:(h2)$) to ["#5" '] ($(h2)!#1!90:(h1)$);
    }

% Basic information
\title{電子情報学専攻 \, 専門 \\ 平成30年 \, 解答・解説}
\author{diohabara}
\date{\today}

\begin{document}
\maketitle

\section*{第1問\ 電気・電子回路}

\section*{第2問\ 計算機アーキテクチャ}
\subsection*{(1)}
\begin{enumerate}
  \item 例外処理が発生する
  \item OSに処理が移る
  \item OSはページテーブルを参照し、LRUページを選択しそのページをディスクに書き戻す
  \item 下記戻したページを置換してアクセスした仮想ページを主記憶上に載せる
\end{enumerate}

\subsection*{(2)}
$2^{48} \divisionsymbol 2^{12} \times 2^3 = 2^3 \times (2^{12})^3 = 512\text{[GB]}$

\subsection*{(3)}
すべてのページについてページテーブルを主記憶に載せるため、あまり参照されないページも含まれており無駄が多い。これにより(2)のように必要な主記憶のサイズが膨大になる。

\subsection*{(4)}
レベル2のページテーブルについては1部のみを主記憶上に載せ、残りは2次記憶に置くことによって、主記憶上のページテーブルが占めるサイズを削減し、無駄に主記憶の容量を占めることがなくなる。

\subsection*{(5)}
\begin{enumerate}
  \item ページテーブルが参照される
  \item 仮想ページアドレスに対応する物理ページアドレスがTLBのエントリに入れられる
  \item TLBに空いているエントリがなければLRUエントリが置換される
\end{enumerate}

\subsection*{(5)}
\subsubsection*{TLBがヒットする際にしないときと比べて主記憶へのアクセス回数が減少する理由}
MMU内にTLBが配置されている場合、TLBがヒットすれば主記憶にアクセスすることなく物理ページアドレスに変換できるから。TLBミスが起こると主記憶上のページテーブルにアクセスする必要が生じる。

\subsubsection*{マルチレベルページテーブルの主記憶へのアクセス回数の変化}
マルチレベルのページテーブルを用いるとページテーブルを参照する際に階層の数だけアクセス回数が増える。

\section*{第3問\ アルゴリズムとデータ構造}
\subsection*{(1)}
\subsection*{(2)}
\subsection*{(3)}
\subsection*{(4)}
\subsection*{(5)}

\section*{第4問\ ネットワーク}

\section*{第5問\ 信号処理}
\subsection*{(1)}
\subsection*{(2)}
\subsection*{(3)}
\subsection*{(4)}

\end{document}
